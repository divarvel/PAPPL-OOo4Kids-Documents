\documentclass[a4paper,11pt]{article}

\usepackage[utf8]{inputenc}
\usepackage[T1]{fontenc}
\usepackage{lmodern}
\usepackage{graphicx}
\usepackage{makeidx}
\usepackage{index}
\usepackage{url}
\usepackage{float}
\usepackage[dvipsnames,usenames]{color}
\usepackage{xcolor}
\usepackage{listings}
%\usepackage[frenchb]{babel}

\lstloadlanguages{C++}


% Informations sur le document.
\title{Rapport Final - PAPPL} \author{Clément \textsc{Delafargue}\\Benjamin \textsc{Vialle}} \date{\today}

% Indentation sur le premier paragraphe d'un chapitre ou section.
%\usepackage{indentfirst}

% Réglage des marges
\usepackage{geometry}
\geometry{left=2cm, right=2cm, top=2.5cm, bottom=2.5cm}

% Mise en forme des en-têtes et pieds de page.
\usepackage{fancyhdr}
\pagestyle{fancy}
\setlength{\headheight}{1.5cm}
\setlength{\textheight}{23cm}
\setlength{\footskip}{2cm}
\lhead{Final Report - PAPPL}
\rhead{\leftmark}
\lfoot{\includegraphics[height=1.5cm]{images/ECN.pdf}}
\rfoot{\includegraphics[height=1.5cm]{images/OOo4Kids.png}}
\fancyfoot[C]{}
\renewcommand{\headrulewidth}{0.4pt}
\renewcommand{\footrulewidth}{0.4pt}

% PDF hyper-linking (set colors to black for printing)
\usepackage[colorlinks]{hyperref}
\usepackage[figure,table]{hypcap}
\hypersetup{
	bookmarksnumbered,
	pdfstartview={FitH},
	citecolor={blue},
	linkcolor={black},
	urlcolor={black},
	pdfpagemode={UseOutlines}
}
\makeatletter
\newcommand\org@hypertarget{}
\let\org@hypertarget\hypertarget
\renewcommand\hypertarget[2]{%
  \Hy@raisedlink{\org@hypertarget{#1}{}}#2%
} 
\makeatother 

%Pour les sauts de ligne entre les paragraphes
\setlength{\parskip}{6pt}

%Supprimer l'alinea
\parindent=0pt

%Pour l'index à la fin du rapport
\makeindex

% Début du document.
\begin{document}

% Définition de la page de garde.
	\begin{titlepage}

		% Pas d'en-tête et pied de page ni de numérotation de page.
		\thispagestyle{empty}

		% Logo Centrale
		\begin{flushleft}
			\includegraphics[scale=0.8]{images/ECN.pdf}
		\end{flushleft}

		\vfill
		
		\begin{center}
			{\Large \textsc{École Centrale de Nantes}} \\

		\vspace{1cm}

		%Insertion d'un titre entouré de traits horizontaux.
		\begin{tabular}{p{0cm} c}
			\hline
			& \\
			& {\huge {\bfseries Final Report - PAPPL}} \\
			& \\
			\hline
		\end{tabular}

		\vspace{2.5cm}

		\texttt{{\large Application Project - OpenOffice4Kids}} \\
		\texttt{{\large January, 2011 - March 2011}} \\

		\vspace{2.5cm}

		% Insertion de deux noms		
		\begin{tabular}{l c r}
			\large{\emph{Author:}} & \hspace{5cm} & \large{\emph{Supervisor:}} \\
			\large{Clément \textsc{Delafargue}} & \hspace{5cm} & \large{Éric \textsc{Bachard}}\\
			\large{Benjamin \textsc{Vialle}} & \hspace{5cm} & \large{Morgan \textsc{Magnin}}\\
		\end{tabular}

		\end{center}

		\vfill

		% Insertion logo Rails
		%\begin{flushleft}
		%	\includegraphics[scale=0.3]{Ruby_on_Rails_logo.jpg}
		%\end{flushleft}
		
		\begin{tabular}{l c r}
				 \includegraphics[scale=0.4]{images/logo_EducOOo.jpg}  & \hspace{6cm} & \includegraphics[scale=0.4]{images/OOo4Kids.png}  \\
		\end{tabular}
		
		% Insertion logo MarkUs
		%\begin{flushright}
		%	\includegraphics[scale=2]{markus.png}
		%\end{flushright}
		
	\end{titlepage}


\newpage
\tableofcontents
% Il faut compiler plusieurs fois pour obtenir une table des matière correcte.

\newpage

% Introduction sur une page, section non numérotée et intégrée à la table des matières.
\setcounter{page}{1}
\rhead{\Large{\textsc{Remerciements}}}
\fancyfoot[C]{\thepage}
\section*{Remerciements}
\addcontentsline{toc}{section}{Remerciements}
Remerciemements particuliers à \emph{Karen Reid}, \emph{Mike Conley},
\emph{Severin Gehwolf} pour leur aide tout au long du projet.

Merci à \emph{Nelle Varoquaux} pour ses retours sur notre code et ses revues
de code sur Review Board.

Merci aux professeurs de l'École Centrale de Nantes pour nous avoir accompagné
tout au long du projet en nous faisant des retours sur l'utilisation de MarkUs
:  \emph{Morgan Magnin}, \emph{Guillaume Moreau}, \emph{Myriam Servières},
\emph{Vincent Tourre}.

\newpage

\rhead{\Large{\textsc{Introduction}}}
\fancyfoot[C]{\thepage}
\section*{Introduction}
\addcontentsline{toc}{section}{Introduction}

\subsection*{Contexte}
\addcontentsline{toc}{subsection}{Context}



%Conclusion sur une page non numérotée et intégrée à la table des matières.
\newpage
\rhead{\Large{\textsc{Conclusion}}}
\section*{Conclusion}
\addcontentsline{toc}{section}{Conclusion}



\newpage
\rhead{\Large{\textsc{Images}}}
\addcontentsline{toc}{section}{List of images}
\listoffigures

\newpage
%\printindex
\rhead{\Large{\textsc{Index}}}
\fancyfoot[C]{\thepage}
\addcontentsline{toc}{section}{Index}




\end{document}
